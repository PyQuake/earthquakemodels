Git\+Hub code source\+:

\href{https://github.com/PyQuake/earthquakemodels}{\tt https\+://github.\+com/\+Py\+Quake/earthquakemodels}

Here is the code for generating Earthquake Risk Models using Genetic Algorithms.

\section*{Some Background}

Some information outside the project frenquently used can be found at\+:

\href{http://www.corssa.org/articles/themevi/zechar}{\tt http\+://www.\+corssa.\+org/articles/themevi/zechar}

\href{http://www.cseptesting.org/}{\tt http\+://www.\+cseptesting.\+org/}

\href{https://github.com/stat157/background/issues/26}{\tt https\+://github.\+com/stat157/background/issues/26}

\href{http://www.jstor.org/stable/3085650?seq=2#page_scan_tab_contents}{\tt http\+://www.\+jstor.\+org/stable/3085650?seq=2\#page\+\_\+scan\+\_\+tab\+\_\+contents}

\subsection*{List of packages }

The project uses the programming language Python, version 3.\+X.\+X. The python libraries used are\+: sys math array numpy datetime time random deap

\subsection*{Getting Started }

\section*{Get the latest version}

To get the last version you should clone the repository from git\+Hub to your local files. 
\begin{DoxyCode}
git clone https://github.com/PyQuake/earthquakemodels.git

# How is the code organized
Most of the code were develop to be run in a python3 interpreter. 
You may find that some files can be bash executed.
In most cases, you can call the most important methods in two ways. 
The direct approach and the self-structured way. 

What we mean from direct approach: It is possible to simply call, 
for example, the method to genarate a model by GA. That method needs
some arguments and you would have to provide them to be able to run the method. 

The self-structured way is an aprroach that allows you to call the
method in the right sequence making it easier to run the methods.
\end{DoxyCode}


\# Executing the main GA methods -\/ self-\/structured way 
\begin{DoxyCode}
in applyGaModel.py you can generate some different kinds of GA models.
All similar methods follows the same parttern: 
  in case you want to create the GA model based on some ETAS ideas,
  the you should run the following method:
    execEtasGaModel(year, times, save=False)
  you have to specify the year for the model and how many times you
  would like to execute the GA. Also, you may chose to save the model
  in a file. But first, you have to create the reference model for
  comparison (in most cases it is the resulting data after filtering
  the catalog by the year) by calling:
    createRealModelforEtas(year, save=False)
  with the same parameters, but times.
\end{DoxyCode}


\#\+Executing the main GA methods -\/ The direct approach 
\begin{DoxyCode}
in the models.etasGaModelNP you can call the GA model method by itself.
You may do it by:
  gaModel(NGEN,CXPB,MUTPB,modelOmega, year, n=500)
in which NGEN, CXPB, MUTPB and n are the GA parameters. modelOmega
is the reference model to be comapared (in most cases it is the
resulting data after filtering the catalog by the year). You should
choose the year parameter as well.
\end{DoxyCode}


\#\+Executing the the testing methods Some tests were implemented. For those, the same idea above is applyed, we both have the The direct approach and the self-\/structured way. But in this case, the self-\/structured way is recommended. This may be out of date.


\begin{DoxyCode}
in applyTests, it is possible to call the tests in two ways. The first
one, execTests(year), executes all tests available for a group of
comparesing models. Or it is possible to run the tests by itself as in
execGamblingScore(year).
\end{DoxyCode}


You may run he tests by their methods by themselves and all of them are or in ./testing\+Alarm\+Based/ or in ./loglikelihood/ 